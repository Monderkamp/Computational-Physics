\documentclass[12pt,a4paper]{article}	%es gibt noch andere Arten of course, wie book zB, aber mit article kommst du gut hin
\usepackage[utf8]{inputenc}
\usepackage[german, english]{babel} 
\usepackage[T1]{fontenc}
\usepackage{amsmath}
\usepackage{bm}
\usepackage{mathtools}

\usepackage{amsfonts}
\usepackage{amssymb}
\usepackage{makeidx}
%\usepackage{esvect}
\usepackage{hyperref}
\usepackage{amsmath}
\usepackage{listings}
\usepackage{subfigure}		%nützlich, wenn du mehrere Bilder in eine Bildumgebung packen möchtest
\usepackage{tabularx}		%für tabellen
\usepackage{chngcntr}
\counterwithin{equation}{section}	%damit Gleichung pro section durchgezählt werden und nicht insgesamt
\counterwithin{figure}{section}		%das gleiche für Bilder
\usepackage{graphicx}
%\usepackage{stackengine}
\usepackage[left=3cm,right=3cm,top=3cm,bottom=3cm]{geometry}
\usepackage[singlespacing]{setspace}	%macht einziligen Zeilenabstand. ansonsten onehalfspacing oder doublespacing

\newcommand{\etan}{\eta_{0}}
\newcommand{\R}{\vv{R}=X\vv{e_{x}}+Y\vv{e_{y}}}
\newcommand{\phid}{\dot{\phi}}
\newcommand{\Yd}{\dot{Y}}
\newcommand{\Xd}{\dot{X}}
\newcommand{\F}{\mathcal{F}}
\newcommand{\p}{\mathcal{P}}
\newcommand{\x}{\textbf}
\newcommand{\norm}[1]{\left\lVert#1\right\rVert}
\newcommand{\abs}[1]{\left\lvert#1\right\rvert}

\begin{document}
%\setcounter{section}{2}

\setlength{\parindent}{0pt}


\thispagestyle{empty}

\begin{titlepage}
	\centering
	\includegraphics[width=0.45\textwidth]{logo_sw.jpg}\par\vspace{1cm}
	\vspace{1cm}
%	{\scshape\Large Spezialisierung\par}
	\vspace{1.5cm}
%	{\LARGE\bfseries working title\\wedge confinement in LC systems  \par}
	{\LARGE\bfseries Assignment Sheet Nr. 6\\  \par}
	\vspace{1cm}
	
	{\large	Paul Monderkamp, Matr.Nr. 2321677\par}
	\vfill
	

	\vfill

% Bottom of the page
	{\large  monderkamp@thphy.uni-duesseldorf.de \par}
	\vspace{2cm}
	%{\large Abgabedatum: 00.04.2017 \par}
\end{titlepage}

\thispagestyle{empty} %macht dass da keine Seitennummer drauf ist.
\newpage	%beendet die Seite und macht auf der nächsten weiter

%\section*{Abstract}

%\newpage


\tableofcontents %macht ein Inhaltsverzeichnis #magic.
\thispagestyle{empty}
\newpage


\section{Exercise 8.1}


\subsection{a)}

        Since no particles are infinitesimaly close to one another, the radial distribution function should go to zero in the limit r -> 0.

        What the computation of the radial distribution function via numerical methods basically consists of is counting the number of particles
        that are at a certain distance to one another and creating a histogram as a function of r. 

        However due to the fact that the number of particles in $[r , r + dr]$ scales with the corresponding area one normalizes with an isotropic 
        distribution which is given by $area*(N/A_total)$.

        Given for instance a square grid on which the particles reside:
        For small r the possible lie relatively far apart and one sees the distinctive peaks in the structure of g(r).
        For r to infinity however the possible r within [r , r + dr] get infinitely close to each other. This results in the physical phenomenon 
        that the different bins in the histogram get similar. Due to the normalization to isotropic density, this results in the limit g(r) -> 1
        for r to $\inf$. 
        In statistical physics this can be interpreted as particles' positions being uncorrolated for infinite distances. 

\subsection{b)} 
$        View MD_functions.cpp "vouble rdf(particle* p, int N, const double BoxL, const int Nbins)" //where vouble refers to vector<double>$

\subsection{c)} 

        i)
            The code from which the function rdf(particle* p, int N, const double BoxL, const int Nbins) is called is located in the file $"distr_test.cpp".$
            View \x{random_conf_g_Nbins=500.png} for the graph.
            A random configuration corresponds to isotropic distribution. Due to the normalization all values of g(r) are equal to one. 
            See above. 
        ii)
            The square lattice configuration is taken from the initial configurations of the previous exercises. 
            The code from which the function rdf(particle* p, int N, const double BoxL, const int Nbins) is called is located in "square_lattice.cpp".
            The particles reside on a square grid with a lattice constant of a = 7/6.
            The analytical results for the possible distances and thus peaks in g(r) can be found via Pythagorean theorem.
            View "square_lattice.png" which shows the first 18 peaks from the measurement of g(r) where the first five peaks are 
            calculated analytically. It can be observed that the results of the analytical calculations coincide with the peaks in the
            measurement of g(r). 
        
\subsection{d)}
        The code that calls the function above for this exercise lies inside "nose_hoover.cpp".
        The radial distribution function from the system that is integrated via the Nosé-Hoover thermostat can be observed in "g_nose_hoover.png". 
        The sampling begins at tau = 1.0; 
        "ths_en_equil.png" shows the diagram of the energies and it can be observed that the system is well equilibrated beyond tau = 1.0;

\section{8.2}
    a)-c) 
        For the function that is used to calculated the MSD, view "double MSD(particle* p0, particle* p, int N, const double BoxL)"
        inside "MD_functions.cpp". The time average is replaced with an ensemble average over the particles using the ergodicity hypothesis. 
        The code for the MD simulation which calls the function above lies within as8.cpp
        The plot for the mean squared displacement can be observed in "MSD.png" and the MSD/t² inside "MSD_over_t_squared.png".
        The ballistic regime is located where the mean squared displacement is proportional to t² and thus where MSD/t² is constant. 
        This coincides with the regime where t \in [0, 0.05]. Observing the graph in "MSD.png" it is visible that MSD(t) is of degree one beyond that point. 



%\section{Appendix}


%\section{References}
%%\begin{enumerate}
%
%%\item i suppose it would be reasonable to give a justification for the low reynolds number and attach a source. 
%
%%%\item das hier ist ne
%%\item durchnummerierte Aufzählung.
%%\item bietet sich natürlich für die Quellen an.
%%\end{enumerate}
%%Wenn du da noch eckige Klammern drum haben willst:\\
%\renewcommand{\labelenumi}{[\theenumi]}
%\begin{enumerate}
%
%\item item 1
%URL: \\
%\url{url1} \\
%published as: item1, last visited: date


%\end{enumerate}


\newpage

%\section{Statutory Declaration}
%
%I declare that I have developed and written the enclosed thesis completely by
%myself, and have not used sources or means without declaration in the text. Any
%thoughts from others or literal quotations are clearly marked. The thesis was
%not used in the same or in a similar version to achieve an academic grading or is being
%published elsewhere.
%\\
%\\
%\\
%\\
%\underline{\hspace{7cm}}\\
%Date, Paul Monderkamp

\end{document}
